\documentclass{scrartcl}

\usepackage[utf8]{inputenc}
\usepackage{amsfonts,amsmath,amssymb}

\title{Lösung zu Blatt 3}
\author{Thomas Schaub}

\begin{document}

\maketitle

\section{Aufgabe 5}

\subsection{A}

\begin{align*}
det(A) = 12 \\
det(A_1) = 215 \\
det(A_2) = 26 \\
det(A_3) = 6
\end{align*}

Alle Hauptminoren sind positiv $\Rightarrow$ Die Matrix ist positiv definit.

\subsection{B}

\begin{align*}
det(B) = -128 \\
det(B_1) = 52 \\
det(B_2) = 20 \\
det(B_3) = -3
\end{align*}

Die Hauptminoren lassen keine Aussage über die Definitheit zu. Also müssen die
Eigenwerte berechnet werden.

TODO

\end{document}
