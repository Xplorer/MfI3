\documentclass{scrartcl}

\usepackage[utf8]{inputenc}
\usepackage{amsfonts,amsmath,amssymb}

\title{Lösung zu Blatt 2}
\author{Thomas Schaub}

\begin{document}

\maketitle

\section{Aufgabe 5}

\subsection{f}

$f\prime$ und $F = \int \Vert f\prime(t) \Vert \, dt$ vom letzten Blatt.

\begin{align*}
\phi(t_1) = \int_0^{t_1}  \Vert f\prime(t) \Vert \, dt = F(t_1) - F(0)
= \sqrt{r^2 + c^2} \cdot t_1 \\
\psi = \phi^{-1} \\
\psi(t) = \frac{1}{\sqrt{r^2 + c^2}} \cdot t \\
g(t) = (f \circ \psi)(t) = (
r \cdot \cos(\frac{t}{\sqrt{r^2 + c^2}}),
r \cdot \sin(\frac{t}{\sqrt{r^2 + c^2}}),
\frac{ct}{\sqrt{r^2 + c^2}}
)
\end{align*}

\subsection{g}

\begin{align*}
\phi(t_1) = 4R(1-\cos(\frac{t_1}{2}))
\end{align*}

Umkehren:

\begin{align*}
x = 4R(1-\cos(\frac{y}{2})) \\
\frac{x}{4R} = 1 - \cos(\frac{y}{2}) \\
\cos(\frac{y}{2}) = 1 - \frac{x}{4R} \\
y = 2 \cdot \cos^{-1} (1 - \frac{x}{4R})
\end{align*}

Daraus folgt

\begin{align*}
g(t) = R(\psi(t) - \sin(\psi(t)), 1 - \cos(\psi(t)))
\end{align*}

\end{document}
